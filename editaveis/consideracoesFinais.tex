\chapter{Considerações Finais}

Neste capítulo serão apresentadas considerações referentes ao que foi realizado
neste trabalho até o presente momento e as expectativas e metas para a próxima
fase.

  As seções estão dispostas em:

  \begin{enumerate}
    \item \textbf{Primeira entrega parcial} considerações acerca dos resultados
    entregues na primeira etapa do respectivo trabalho, referente ao TCC1.
    \item \textbf{Entrega final} considerações acerca das expectativas almejadas
    para a realização da segunda etapa do respectivo trabalho, referente ao TCC2.
    \item \textbf{Cronograma} Distribuição planejada das fases de execução deste
    trabalho durante o período subsequente.
  \end{enumerate}

\section{Primeira entrega parcial}
\section{Entrega final}
\section{Cronograma}

Esta seção visa apresentar o cronograma para desenvolvimento do presente trabalho. Este
cronograma está dividido em duas etapas: Primeira Entrega Parcial (TCC1) e Entrega Final
(TCC2).

O cronograma referente a Primeira Entrega Parcial é apresentado no \autoref{quad:CronogramaTCC1}.
Em relação a atividade de \textit{Pesquisa e definição da metodologia}, foi definido o
\nameref{fig:ProcessoAnaliseComparativa} apresentado na \autoref{fig:ProcessoAnaliseComparativa}
que será aplicado para elencar quais métricas e informações serão coletadas durante a
\nameref{sec:Etapa2}.  Contudo, a execução desse processo irá definir aspectos como métricas
e tecnologias que ainda não estão descritos na \nameref{sec:Metodologia}. Por este motivo,
esta atividade foi considerada como parcialmente concluída.

\begin{quadro}
    \caption{Cronograma: primeira entrega parcial\label{quad:CronogramaTCC1}}
    \begin{tabular}{ | m{4cm} | c | c | c | c | m{2.5cm} | }
    \hline
    \textbf{Atividade} &
        \textbf{Agosto} &
        \textbf{Setembro} &
        \textbf{Outubro} &
        \textbf{Novembro} &
        \textbf{Status} \\ \hline
    Levantamento e definição do tema &
        X &
        X &
        &
        &
        Concluído \\ \hline
    Definição do objetivo geral e dos específicos &
        &
        X &
        &
        &
        Concluído \\ \hline
    Pesquisa e levantamento bibliográfico &
        &
        X &
        X &
        &
        Concluído \\ \hline
    Pesquisa e definição da metodologia &
        &
        &
        X &
        &
        Parcialmente concluído \\ \hline
    Análise inicial da arquitetura &
        &
        &
        &
        X &
        Concluído \\ \hline
    \end{tabular}
\end{quadro}
