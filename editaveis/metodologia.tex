\chapter{Metodologia}

Neste capítulo serão apresentados os métodos, ferramentas e o processo definido
para aplicação no presente trabalho visando alcançar o seu objetivo final.

  As seções estão dispostas em:

  \begin{enumerate}
    \item \textbf{Metodologia de Desenvolvimento} Descrição do fluxo que será
    adotado para análise dos fatores pertinentes ao escopo do presente trabalho.
    \item \textbf{Tecnologias e ferramentas} Breve descrição das tecnologias,
    que serão utilizados.
    \item \textbf{Cronograma} Distribuição planejada das fases de execução deste
    trabalho durante o período subsequente.
  \end{enumerate}

\section{Metodologia de Desenvolvimento}

Pretende-se com estes trabalho realizar uma análise comparativa entre duas
arquiteturas de software diferentes adotadas dentro da empresa Rua Dois para um
mesmo sistema. Dessa forma, esta seção visa descrever quais serão os objetos de
análise e como se darão as etapas de coleta de dados e análise.

\subsection{Objetos de análise}

Visto que presente trabalho refere-se a uma análise sobre o software da Rua Dois
dentro do contexto de uma \textit{startup}, optou-se por definir duas fases arquiteturais
marcantes dentro da empresa como objetos de estudos para o presente trabalho. A composição
de cada uma dessas fases será abordada na seção \ref{sec:ArquiteturaDoSistema}, a
seguir é apresentada uma breve descrição sobre cada uma dessas fases:

    \begin{description}
        \item [Fase 1] Fase inicial de desenvolvimento do software da Rua Dois,
        marcado por uma arquitetura de microsserviços e um forte período de validação
        de ideias.
        \item [Fase 2] Segunda fase de desenvolvimento do software da Rua Dois,
        marcado por uma arquitetura monolítica e um escopo melhor definido.
    \end{description}

\subsection{Coleta e levantamento de dados}

Para cada objeto de análise definido serão levantados dados e informações julgadas
como pertinentes para a análise final. Abaixo segue a descrição e o objetivo de
cada tópico que deverá ser observado.

    \begin{description}
        \item [Qualidade de código] Usando métricas de qualidade de código pretende-se
        avaliar questões referentes a coesão e acomplamento do sistema em cada uma das
        fases.
        \item [Práticas de desenvolvimento de Software] Elencar quais práticas de
        desenvolvimento de software foram aplicadas em cada fase e como estas práticas
        afetaram a qualidade do sistema em questão.
        \item [Nível de conhecimento e experiência da equipe no contexto de \textit{startups}]
        Aplicação de um questionário entre os membros do time de desenvolvimento participantes
        de cada fase, com o intuito de avaliar o domínio do time com as tecnologias utilizadas
        e a experiência com desenvolvimento de software dentro de \textit{startups}.
        \item [Tecnologias utilizadas] Identificar quais foram as tecnologias utilizadas em
        cada fase, como elas influenciaram o desenvolvimento e qual a escalabilidade de cada
        tecnologia.
        \item [Escalabilidade do sistema] Por meio de testes de escalabilidade, coletar
        informações sobre o nível de escalabilidade da arquitetura adotada em cada fase.
        \item [Relação com o ciclo de vida de uma \textit{startup}] Avaliar qual foi o
        contexto de desenvolvimento com base nas fases de uma \textit{startup} e como a
        arquitetura adotada influênciou nesse contexto.
    \end{description}

\subsection{Análise dos dados}

Uma vez coletado os dados da etapa anterior, pretende-se fazer uma análise comparativa
entre as fases definidas como objeto de estudo, visando alinhar as diferenças de
escalabilidade e de desenvolvimento referentes a cada uma. Com base nessa análise
comparativa pretende-se avaliar a possibilidade de orientar um desenvolvimento
sustentável e escalável dentro da Rua Dois, elencando possíveis medidas a serem
tomadas dentro da empresa para tal.

\section{Ferramentas}

\subsection{Google Forms}

\section{Cronograma}
