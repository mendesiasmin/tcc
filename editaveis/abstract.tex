\begin{resumo}[Abstract]
 \begin{otherlanguage*}{english}
  In the last years there was a growing emergence of companies called startups.
  These companies aim to seek, as soon as possible, a repeatable and scalable
  business model that creates market innovation. In this context was born the
  startup Rua Dois with the purpose of innovating the real estate market. 
  However, during their quest for innovation and scalability the software
  development process was disturbed. Leading to some technical decisions that
  made it difficult for software to evolve at the same speed as the company wanted.
  Consequently, it was necessary to simplify the entire architecture of the
  software developed in order to better adapt to the current startup context. The
  present work will perform a comparative analysis between the two architectures
  adopted in the company in order to evaluate them and define possible guidelines
  that guide the Rua Dois regarding their expectations of scalable software
  development.
   \vspace{\onelineskip}
 
   \noindent 
   \textbf{Key-words}: Case study. Scalability. Software development. Startup.
 \end{otherlanguage*}
\end{resumo}
