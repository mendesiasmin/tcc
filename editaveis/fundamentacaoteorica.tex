\chapter[Fundamentação Teórica]{Fundamentação Teórica}

Neste capítulo será apresentado as bases teóricas que permeiam o escopo
do presente trabalho. Partindo do entendimento sobre o contexto de
\textit{startups} até as questões relacionadas a escalabilidade de softwares.
As seções estão dispostas em:

  \begin{description}
    \item[\textit{Startups}:] contexto, organização e propósitos de uma 
      \textit{startup}.
    \item[Escalabilidade de Software:] definição de Escalabilidade e 
      possíveis formas de aplicação.
  \end{description}

\section{\textit{Startups}}

No livro \textit{A Revolução das Startups} de \citeonline{ARevolucaoDasStartups},
o autor descreve uma Geração Y, nascida entre 1980 e 2000, detentora de um
espírito jovem em busca de ser feliz fazendo algo que seja realmente
impactante para a sociedade. Diferente das gerações passadas, a Geração Y
possui em suas mãos a Internet e o conhecimento para utilizar esta
ferramenta em todo o seu potencial.

Essa geração leva a sociedade a um contexto no qual as pessoas estão sempre
se questionando se existe outra forma de fazer, se podemos fazer melhor e
se somos capazes de resolver algum problema \cite{ARevolucaoDasStartups}.
A partir dessas indagações nascem \textit{startups} revoluniciárias, como
o Waze\footnote{Empresa que conecta motoristas visando melhorar o tráfego
dentro das cidades. Saiba mais em: \url{https://www.waze.com}}, lançado em
2009 com a proposta de mudar a forma das pessoas se locomoverem, deixando 
de lado a maneira tradicional de se usarem mapas e fazendo com que as 
pessoas interajam e contribuam para o próprio mapeamento
\cite{NepomucenoSucessoDoWaze}.

\subsection{O que é uma \textit{Startup}?}

  \begin{quotation}
  "Uma \textit{startup} é uma organização formada para buscar um modelo de 
  negócios repetível e escalável." \cite{SteveBlankFirstPrinciples}
  \end{quotation}

A definição apresentada por Steve Blank traz características
fundamentais para entender o conceito de uma \textit{startup}. Começando
pela organização que representa um grupo de pessoas alinhadas em prol
de um mesmo objetivo. Esse grupo de pessoas juntamente com seus ideais
são a essência de uma \textit{startup}, afinal, são elas que fazem
o negócio fluir somando as habilidades individuais de cada um
\cite{ARevolucaoDasStartups}.

O modelo de negócio é a descrição de como a empresa cria, entrega e captura
valor, exibindo todos os fluxos entre as diferentes partes da empresa,
como o produto é distribuído para os clientes, como o dinheiro retorna, a
estruturas de custos e a interação com outras empresas parceiras. Uma
\textit{startup} consiste essencialmente em uma organização criada para
procurar um modelo de negócios, iniciando com uma visão sobre o produto e
uma série de hipóteses sobre o modelo de negócios
\cite{SteveBlankFirstPrinciples}.

Segundo \citeonline{ARevolucaoDasStartups}, ser repetível significa que 
a ideia deve ser facilmente replicável em outras regiões, países ou até
mesmo outros setores, visando diversificar os ganhos. Ou seja, quanto mais
pessoas utilizam o serviço, melhor para as \textit{startups}.

Atingir o objetivo de ser repetível carrega junto a missão de ser também
escalável. Afinal, reproduzir o modelo de negócio em outras regiões e países
também envolve a capacidade da empresa de crescer, preferencialmente de uma
forma saudável. Para tal é necessário que os serviços sejam prestados sem
demandar recursos na mesma proporção que o seu crescimento, usando somente 
uma estrutura básica comum a todos \cite{CassioSpina}.

No livro \textit{A Startup Exuta}, \citeonline{StartupEnxuta} traz outra
definição de \textit{startup} a qual diz: \textit{"Startup é uma instituição
humana projetada para criar novos produtos e serviços sob condições de extrema
incerteza"}. Um contexto no qual toda aprendizagem é válida e deve ser
reutilizada em um ciclo constante de coleta de \textit{feedbacks} dos clientes.

Essa segunda definição, adiciona uma nova característica ao conceito de
\textit{startup} que é o ambiente extremamente incerto no qual essas empresas
estão inseridas. Com base em  \apudonline{Rueda}{GardelinRausp}, esta incerteza
ambiental é perceptível em uma empresa quando há dúvidas, por parte dos gerentes,
quanto a uma série de fatores, entre eles: a viabilidade de futuras tecnologias,
as expectativas de mudanças de consumo e preferências sociais para os produtos
e serviços e possíveis mudanças na legislação.

% TODO Descrever as fases da startup de acordo com Steve Blank
