\chapter[Fundamentação Teórica]{Fundamentação Teórica}

Neste capítulo será apresentado as bases teóricas que permeiam o escopo
do presente trabalho. Partindo do entendimento sobre o contexto de
\textit{startups} até as questões relacionadas a escalabilidade de softwares.
As seções estão dispostas em:

  \begin{enumerate}
    \item \textbf{\textit{Startups}:} contexto, organização e propósitos de uma 
      \textit{startup}.
    \item \textbf{Escalabilidade de Software:} definição de Escalabilidade e 
      possíveis formas de aplicação.
    \item \textbf{Amazon Web Services:} apresentação e descrição dos serviços
    pertinentes para o presente trabalho.
  \end{enumerate}

\section{\textit{Startups}}

No livro \textit{A Revolução das Startups} de \citeonline{ARevolucaoDasStartups},
o autor descreve uma Geração Y, nascida entre 1980 e 2000, detentora de um
espírito jovem em busca de ser feliz fazendo algo que seja realmente
impactante para a sociedade. Diferente das gerações passadas, a Geração Y
possui em suas mãos a Internet e o conhecimento para utilizar esta
ferramenta em todo o seu potencial.

Essa geração leva a sociedade a um contexto no qual as pessoas estão sempre
se questionando se existe outra forma de fazer, se podemos fazer melhor e
se somos capazes de resolver algum problema \cite{ARevolucaoDasStartups}.
A partir dessas indagações nascem \textit{startups} revoluniciárias, como
o Waze\footnote{Empresa que conecta motoristas visando melhorar o tráfego
dentro das cidades. Saiba mais em: \url{https://www.waze.com}}, lançado em
2009 com a proposta de mudar a forma das pessoas se locomoverem, deixando 
de lado a maneira tradicional de se usarem mapas e fazendo com que as 
pessoas interajam e contribuam para o próprio mapeamento
\cite{NepomucenoSucessoDoWaze}.

\subsection[OQueEStartup]{O que é uma \textit{Startup}?}
\label{sec:OQueEUmaStartup}

  \begin{quotation}
  "Uma \textit{startup} é uma organização formada para buscar um modelo de 
  negócios repetível e escalável." \cite{SteveBlankFirstPrinciples}
  \end{quotation}

A definição apresentada por Steve Blank traz características
fundamentais para entender o conceito de uma \textit{startup}. Começando
pela organização que representa um grupo de pessoas alinhadas em prol
de um mesmo objetivo. Esse grupo de pessoas juntamente com seus ideais
são a essência de uma \textit{startup}, afinal, são elas que fazem
o negócio fluir somando as habilidades individuais de cada um
\cite{ARevolucaoDasStartups}.

O modelo de negócio é a descrição de como a empresa cria, entrega e captura
valor, exibindo todos os fluxos entre as diferentes partes da empresa,
como o produto é distribuído para os clientes, como o dinheiro retorna, a
estruturas de custos e a interação com outras empresas parceiras. Uma
\textit{startup} consiste essencialmente em uma organização criada para
procurar um modelo de negócios, iniciando com uma visão sobre o produto e
uma série de hipóteses sobre o modelo de negócios
\cite{SteveBlankFirstPrinciples}.

Segundo \citeonline{ARevolucaoDasStartups}, ser repetível significa que 
a ideia deve ser facilmente replicável em outras regiões, países ou até
mesmo outros setores, visando diversificar os ganhos. Ou seja, quanto mais
pessoas utilizam o serviço, melhor para as \textit{startups}.

Atingir o objetivo de ser repetível carrega junto a missão de ser também
escalável. Afinal, reproduzir o modelo de negócio em outras regiões e países
também envolve a capacidade da empresa de crescer, preferencialmente de uma
forma saudável. Para tal é necessário que os serviços sejam prestados sem
demandar recursos na mesma proporção que o seu crescimento, usando somente 
uma estrutura básica comum a todos \cite{CassioSpina}.

No livro \textit{A Startup Exuta}, \citeonline{StartupEnxuta} traz outra
definição de \textit{startup} a qual diz: \textit{"Startup é uma instituição
humana projetada para criar novos produtos e serviços sob condições de extrema
incerteza"}. Um contexto no qual toda aprendizagem é válida e deve ser
reutilizada em um ciclo constante de coleta de \textit{feedbacks} dos clientes.

Essa segunda definição, adiciona uma nova característica ao conceito de
\textit{startup} que é o ambiente extremamente incerto no qual essas empresas
estão inseridas. Com base em  \apudonline{Rueda}{GardelinRausp}, esta incerteza
ambiental é perceptível em uma empresa quando há dúvidas, por parte dos gerentes,
quanto a uma série de fatores, entre eles: a viabilidade de futuras tecnologias,
as expectativas de mudanças de consumo e preferências sociais para os produtos
e serviços e possíveis mudanças na legislação.

\subsection{Fases de uma Startup}
% TODO Descrever as fases da startup de acordo com Steve Blank

\section{Escalabilidade}

Segundo \citeonline {PlatformEcosystems}, escalabilidade refere-se ao grau em que
o desempenho funcional e financeiro de um subsistema é independente de seu tamanho.
Podendo manter seu desempenho e função retendo todas as propriedades desejadas sem
um aumento correspondente em sua complexidade interna
\apud{EngineeringSystems}{PlatformEcosystems}. Para Tiwana, isso implica em duas
proposições pertinentes sobre a escalabilidade: a primeira é de que escalabilidade não
se refere exclusivamente a aumentar a escala de um sistema, como habitualmente
pensamos, mas também a capacidade deste sistema de se contrair conforme a necessidade.
A segunda proposição diz que o desempenho de um sistema escalável pode significar
tanto a sua capacidade técnica de desempenho quanto a sua capacidade financeira.

Assim, a escalabilidade reflete a capacidade do software em crescer e evoluir
conforme as demandas do usuário. Dessa forma, é importante analisar se o
modelo de negócios da empresa depende do crescimento desse sistema. Uma vez que
não existe essa dependência, a escalabilidade não se torna um requisito básico,
podendo um software não escalável funcionar bem com uma capacidade limitada
\cite{ConceptaScalability}.

No texto \textit{The Importance of Scalability In Software Design}
da \citeonline{ConceptaScalability}, a empresa Concepta
\footnote{Saiba mais em: \url{https://conceptainc.com}} diz:

  \begin{quotation}
    "A escalabilidade é um componente essencial do software corporativo. Priorizá-lo
    desde o início leva a menores custos de manutenção, melhor experiência do usuário
    e maior agilidade."
  \end{quotation}

Nessa abordagem apresentada pela Concepta, a escalabilidade vai além de atender
a um grande volume de demandas de requisições, e passa a fazer parte da
experiência do usuário além de influenciar positivamente a manutenção desse
software. Seguindo essa visão de escalabilidade apresentada pela Concepta,
existem três fatores que afetam significantemente a escalabilidade de softwares:

  \begin{enumerate}
    \item A capacidade de usuários simultâneos ou conexões possíveis.
    Aumentar essas capacidade deve ser tão simples quanto disponibilizar mais
    recursos ao software.
    \item A capacidade de armazenamento, principalmente para aplicações
    que apresentam muitos dados não estruturados. A escolha do tipo do banco de
    dados que será usado e o uso de uma indexação adequada são fatores que
    podem influenciar diretamente a escalabilidade.
    \item O próprio código. Desenvolvedores inexperientes tendem a ignorar
    o fato de que o código deve ser escrito para que possa ser facilmente
    adicionado ou modificado, sem a necessidade de refatorar o código antigo.
    Bons desenvolvedores evitam a duplicação de esforços, reduzindo o tamanho
    e a complexidade geral da base do código.
  \end{enumerate}

Priorizar esses pontos no início da construção do software, traz agilidade e
benefícios que garantem o crescimento saudável do sistema.

\subsection{Bancos NoSQL}

Atualmente, as aplicações têm trabalhado com um volume de dados cada vez maior.
E nesse contexto, as propriedades ACID - \textit{Atomicity, Consistency,
Isolation, Durability} implementadas pelos bancos de dados relacionais,
tornam-se muito onerosas mediante as necessidades específicas desses aplicativos.
Assim, o custo associado ao dimensionamento dos bancos de dados tradicionais em
frente ao crescente volume de dados se torna muito caro \cite{Gajendran}.

Diante desta situação, desenvolvedores buscam soluções alternativas para o
armazenamento de dados, dentre-elas estão os bancos não-relacionais. O termo
NoSQL significa \textit{"not only SQL"}, e refere-se a um conjunto
de banco de dados que não possuem as características tradicionais de um
banco de dados relacional \cite{Gajendran}. Estes bancos são projetados para
atender uma larga escala de armazenamento de dados e processamento paralelo
massivo em cima desses dados \cite{NewEraOfDatabases}.

Para obter essa escalabilidade, os bancos NoSQL \textit{"abriram mão"} de
manter a consistências dos dados, como ocorre nos bancos de dados relacionais,
resultando em sistemas conhecidos como BASE - \textit{Basicamente Disponível,
Estado suave, Eventualmente consistentes}. Estes sistemas não possuem
transações no sentido clássico e introduzem restrições no modelo de dados
para permitir melhores esquemas de partição \apud{SurveyNosql}{NewEraOfDatabases}.

Assim eles oferecem um armazenamento relativamente barato e altamente escalonável
para pequenos pacotes de registros, como \textit{logs}, leitura de medidores,
\textit{snapshots} e para dados semi-estruturados ou não-estruturados, como arquivos
de e-mail, \textit{xml} e documentos. A estrutura distribuída desses bancos também
os tornam ideias para processamento massivo de dados em lote \cite{NewEraOfDatabases}.

\subsection{Sistemas monolíticos}

A Arquitetura Monolítica é um padrão de desenvolvimento de software no qual um aplicativo
é criado com uma única base de código, um único sistema de compilação, um único binário
executável e vários módulos para recursos técnicos ou de negócios. Seus componentes
trabalham compartilhando mesmo espaço de memória e recursos formando uma unidade de
código coesa \cite{NatalliaSakovich}.

Esse padrão arquitetural permite um desenvolvimento fácil por ser de conhecimento
da maioria dos desenvolvedores de software e apresentar baixa complexidade para a
execução de tarefas como \textit{deploy}, confecção de testes e compartilhamento
do código. Contudo, a longo prazo começam a surgir dificuldades para manter essa
arquitetura, entre elas estão:

  \begin{enumerate}
    \item Dificuldade em entender e alterar o código que ao longo do tempo se torna
    muito extenso e coeso.
    \item Limitação na agilidade de atualização do software, uma vez que para cada
    pequena alteração é necessário reimplantar o código por completo.
    \item Necessita de muito esforço para adotar uma nova tecnologia, sendo
    necessário adaptar todo o código da aplicação para a nova ferramenta.
    \item O sistema perde a confiabilidade pois a medida que cresce um \textit{bug}
    em uma única parte do código pode interromper todo o software já que todos os
    componentes estão conectados.
  \end{enumerate}

\subsection{Microsserviços}

Segundo \citeonline{Dragoni2017} microsserviços podem ser definidos como um
processo coeso e independente que interage por meio de mensagens. Isso implica
em componentes conceitualmente independentes e implantados isoladamente utilizando
de recursos dedicados.

\subsection{Como medir escalabilidade?}

\section{AWS}
\subsection{Lambdas}
\subsection{Instâncias EC2}
\subsection{DynamoDB}
