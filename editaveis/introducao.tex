\chapter[Introdução]{Introdução}
% \addcontentsline{toc}{chapter}{Introdução}

Neste capítulo serão abordados os aspectos gerais relacionados ao contexto
do presente Trabalho de Conclusão de Curso. As seções estão dispostas em:

  \begin{enumerate}
    \item \textbf{Contexto:} apresentação do contexto e indagações deste
    trabalho.
    \item \textbf{Justificativa:} descrição sobre a importância de se
    realizar o presente estudo.
    \item \textbf{Objetivos:} descrição dos objetivos gerais e específicos
    que se pretende alcançar com ao final deste trabalho.
    \item \textbf{Organização:} breve descrição sobre a organização deste
    trabalho.
  \end{enumerate}

\section{Contexto}

Existe no mundo um novo cenário emergente no qual, empresas nascentes,
denominadas \textit{startups}, produzem inovação ao ponto de serem capazes
de competir com empresas já estabelecidas no mercado \cite{CapacidadeDeInovacao}.
O ambiente de negócios está mudando devido a fatores como globalização, política
e tecnologia. Neste novo contexto, tudo que se conhece dos negócios são
hipóteses e um constante alto grau de risco \cite{Akiyoshi}.

Nesse cenário, surgiu a \textit{startup} Rua Dois Tecnologias. Com uma série
de hipóteses sobre o mercado imobiliário, a empresa visa inovar esse ramo que
é ainda muito burocrático e carente de melhorias \cite{LouisaXu}. E nessa
euforia por inovação, a \textit{startup} tem em seu modelo de negócios a
constante busca pela validação rápida de ideias mas de uma forma que ela
esteja preparada pra crescer e se expandir rapidamente, conforme os princípios
de repetibilidade e escalabilidade\footnote{Os princípios de Steve Blank sobre
\textit{startups} serão abordados com maior profundidade na seção
\ref{sec:OQueEUmaStartup}.} de \citeonline{SteveBlankFirstPrinciples} para
\textit{startups}.

Nessa busca por escalabilidade a empresa tomou uma série de decisões tecnológicas
baseadas no que seria escalável, mas sem uma análise coerente com o sistema
proposto e com o contexto de validação de ideias que a empresa vivenciava.
A exemplo está a escolha de utilizar uma arquitetura de microsserviços sem conhecer
de fato a demanda dos serviços que seriam prestados, ocasionando dificuldades para
a empresa de modificar e evoluir esse sistema com a rapidez que é desejada para uma
\textit{startup}.

Assim surgem as indagações do presente trabalho. Neste contexto das \textit{startups}
o software produzido é incessantemente alterado em busca da melhor solução para um
determinado problema, é possível preparar esse software de forma que ele possa ser
escalável? Se for possível, quais práticas e recomendações podem ser aplicadas para
alcançar este objetivo?

Com foco nessas questões, este trabalho almeja realizar um estudo de caso sobre o
desenvolvimento de software na \textit{startup} Rua Dois, visando avaliar as questões
apresentadas e direcionar esta empresa para um caminho mais consolidade em relação
aos seus objetivos de crescimento do sistema.

\section{Justificativa}

As metodologias ágeis\footnote{Métodos iterativos para desenvolvimento de software,
que visam a reação a mudanças conforme sejam as necessidades do cliente. Vide
\url{http://agilemanifesto.org/}} visam diminuir os custos com retrabalho mediante as
constantes mudanças de requisitos no processo de desenvolvimento de software,
não apenas acomodando essas mudanças, mas abraçando-as. Assim, o design do
software é construído de forma contínua \cite{AgileSoftwareInnovation}.
Mas como garantir qualidade na construção desse design sendo que esses requisitos
na verdade são hipóteses a serem comprovadas sobre o software? E em paralelo
a validação dessas hipóteses, preparar esse software para atender a grande
demanda que é almejada para as \textit{startups}? \citeonline{StartupEnxuta}
em seu livro \textit{A Startup Enxuta}, traz a seguinte reflexão sobre esta
realidade:

  \begin{quotation}
    "Como sociedade, dispomos de um conjunto comprovado de técnicas para
    administrar grandes empresas, e conhecemos as melhores práticas para
    construir produtos físicos. No entanto, quando se trata de \textit{startups}
    e inovação, ainda estamos atirando no escuro."
  \end{quotation}

Portanto, este trabalho justifica-se por sua contribuição em buscar, neste
cenário citado por Ries como obscuro, compreender as necessidades do desenvolvimento
de software e como essas necessidades podem ser alinhadas a um contexto real
que é o da \textit{startup} Rua Dois.

\section{Objetivos}

\subsection{Objetivo Geral}

Este trabalho tem como objetivo geral realizar um estudo de caso acerca do
desenvolvimento de software na \textit{startup} Rua Dois, contrapondo o contexto de
validação de ideias com as necessidades de escalabilidade durante esse
desenvolvimento. Almejando, ao final, obter práticas e recomendações que possam
ser aplicadas na empresa com o propósito de direcioná-la em relação as suas
expectativas de escalabilidade.

\subsection{Objetivos Específicos}

As seguintes metas foram levantadas visando atingir o objetivo geral do presente
trabalho:

  \begin{enumerate}
    \item Compreender o contexto e as necessidades de uma \textit{startup} em
    relação ao desenvolvimento de software.
    \item Compreender o conceito de escalabilidade e técnicas para aplicação
    da mesma.
    \item Contrapor as questões associadas a escalabilidade com as necessidades
    de uma \textit{startup}.
    \item Compreender a arquitetura de software atual da Rua Dois.
    \item Identificar os principais problemas sobre a atual arquitetura da Rua
    Dois, os quais atrapalham a empresa a se desenvolver como \textit{startup}.
    \item Planejar uma possível solução que prepare o sistema atual para ser
    escalado de uma forma saudável.
    \item Definir meios de monitorar a demanda de recursos desse sistema.
  \end{enumerate}

\section{Organização do Documento}

Este documento foi organizado nos seguintes tópicos:

  \begin{enumerate}
    \item \textbf{Fundamentação Teórica:} conceitualização e exploração de
    aspectos considerados pertinentes para o presente estudo.
    \item \textbf{Metodologia:} descrição da metodologia adotada para
    realização deste estudo.
    \item \textbf{Desenvolvimento:} discussão acerca da arquitetura de
    software adotada na Rua Dois e sobre pontos pertinentes referente a
    \textit{startup} e escalabilidade.
    \item \textbf{Considerações Finais:} resultados das pesquisas mediante a
    primeira etapa deste trabalho.
  \end{enumerate}

